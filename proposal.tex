\documentclass[11pt, letterpaper]{article}
\usepackage{fullpage}

\begin{document}
\title{Proposal}
\author{Harsh Patel, Ayaan Kazerouni}
\date{}
\maketitle

\section*{Description of the Data}
This data set contains information about European soccer.
It contains data about more than 25000 matches and more than 10000 players and their attributes.
This data covers 11 European countries and their championships and spans an 8-year period (2008 - 2016).
The data set provides detailed match events such as goal types, possessions and more for more than 10000 matches.
Player and team attributes are sourced from the popular EA Sports FIFA video game series.


The data is provided in a relational database (RDB) of \textbf{Matches} (115), \textbf{Players} (7), \textbf{Player-Attributes} (42), \textbf{Teams} (5), \textbf{Team-Attributes} (25), \textbf{Leagues} (3), and \textbf{Countries} (2).
Of these, the entities of interest are Player-Attributes, Team-Attributes, and Matches.
We think the other entities will be of use for mostly indexing purposes.

\subsection*{Reasons for Choosing this Dataset}
This dataset is interesting to us because both partners are sports fans.
We also think it would be interesting if we could come up with a model to predict match winners, key player attributes, and league winners.
The provider of the dataset mentioned a match-prediction accuracy of 53\%.
Even though we are unlikely to succeed, we think it would be interesting to try to beat that percentage.

Kaggle provides for iterative improvements of the dataset.
The most recent versions were uploaded within 10 days of this writing (version 7 through 10).
The frequently updated nature of this dataset suggests that care has been taken to maintain the high quality of the data.

\section*{Proposed Data Preprocessing Steps}
A good portion of the data pre-processing and aggregation was done by the provider (Kaggle user Hugo Mathien).

Thanks to Mathien's efforts in collecting and organising this dataset from several sources, we are afforded the ability to issue straightforward SQL queries in our data aggregation steps.
This will enable us to reduce dimensionality to easily focus only on areas of interest.

For our initial problem of predicting match winners, we will focus on the \textbf{Match}, \textbf{Team}, and \textbf{Team-Attributes} entities.
Using SQL, we will extract team data for each match, keeping track of an attribute vector for the winning team.
Based on these 10000+ vectors, for future matches, we could try to predict the outcome given the teams' attributes.
Each match is inherently a comparison between two teams.
We will take advantage of this fact to build a model that, given two teams, can tell which one is likely to win, or if the match will be a draw.
We could also attempt to predict the goal-count of a match, which would have implications on bookkeeping and betting.

For model creation, we are partial to decision trees over rule-based classifiers.
With rule-based classifiers, there is a bit more overhead with ordering and prioritising rules that is not present when using decision trees. 
This overhead comes without an increase in speed of classification. Due to the reduced complexity, we prefer decision trees.

After aggregating data of interest, we will apply dimensionality reduction techniques like Principal Component Analysis to narrow down the dataset to the important features only.

\section*{Proposed Data Mining Approaches}


\end{document}
