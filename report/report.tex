\documentclass[11pt]{article}
\usepackage{fullpage}
\usepackage{hyperref}

\begin{document}
\title{Predicting the Outcome of Soccer Matches Using Logistic Regression}
\author{Ayaan Kazerouni and Harsh Patel}
\date{}
\maketitle

\section{Problem Statement}
Soccer is a widely played sport, and is arguably the only globally played sport.
We thought it would be interesting to develop a model that could predict the winner of a soccer game, a task called the `holy grail' by the provider of the dataset.
Since the data was provided as a relational database, significant pre-processing steps were undertaken before modelling began, described in Section~\ref{sec-preprocessing}.

\section{Data Description}
We have chosen a European soccer dataset~\cite{mathien1} that contains data about more than 25000 soccer matches and more than 10000 players and their attributes.
Covering 11 European countries and their championships, this dataset spans an 8-year period (2008 - 2016).
It provides detailed match events such as goal types, possessions and more, for more than 10000 matches.
Player and team attributes are sourced from the popular EA Sports FIFA video game series.

The data is provided in a relational database (SQLite3 Database) of \textit{Matches}, \textit{Players}, \textit{Player-Attributes}, \textit{Teams}, \textit{Team-Attributes}, \textit{Leagues}, and \textit{Countries}.
A description of the magnitude of the data is provided in Table~\ref{tab:data-desc}.

\begin{table}[ht]
\centering
\begin{tabular}{|c|c|c|}
\hline
\textbf{}                   & \textbf{Count} & \textbf{Dimensions} \\ \hline
\textbf{Match}              & 25979          & 115                 \\ \hline
\textbf{Teams}              & 299            & 5                   \\ \hline
\textbf{Team\_Attributes}    & 1458           & 25                  \\ \hline
\textbf{Player}             & 11060          & 7                   \\ \hline
\textbf{Player\_Attributes} & 183978         & 42                  \\ \hline
\textbf{Country}            & 11             & 2                   \\ \hline
\textbf{League}             & 11             & 3                   \\ \hline
\end{tabular}
\caption{Magnitude of the available data}
\label{tab:data-desc}
\end{table}

With our objective in mind, the following two tables were of interest to us:
\begin{itemize}
  \item Match
  \item Team\_Attributes
\end{itemize}

Looking at Table~\ref{tab:data-desc} again, notice that Match and Team\_Attributes have 115 and 25 attributes, respectively.
The Match table contained the following relevant dimensions:
\begin{itemize}
  \item season
  \item stage
  \item date
  \item home_team_id
  \item away_team_id
  \item home_team_goal
  \item away_team_goal
  \item goal
  \item shoton
  \item shotoff
  \item cross
  \item corner
  \item possession
\end{itemize}


\bibliography{report.bib}
\bibliographystyle{plain}
\end{document}
